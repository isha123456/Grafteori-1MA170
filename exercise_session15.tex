\documentclass[nobib]{tufte-handout}

\title{Exercise session 15: Extremal graph theory and Szemerédi's regularity lemma $\cdot$ 1MA020}

\author[Vilhelm Agdur]{Vilhelm Agdur\thanks{\href{mailto:vilhelm.agdur@math.uu.se}{\nolinkurl{vilhelm.agdur@math.uu.se}}}}

\date{29 November 2023}


%\geometry{showframe} % display margins for debugging page layout

\usepackage{graphicx} % allow embedded images
  \setkeys{Gin}{width=\linewidth,totalheight=\textheight,keepaspectratio}
  \graphicspath{{graphics/}} % set of paths to search for images
\usepackage{amsmath}  % extended mathematics
\usepackage{booktabs} % book-quality tables
\usepackage{units}    % non-stacked fractions and better unit spacing
\usepackage{multicol} % multiple column layout facilities
\usepackage{lipsum}   % filler text
\usepackage{fancyvrb} % extended verbatim environments
  \fvset{fontsize=\normalsize}% default font size for fancy-verbatim environments

\usepackage{color,soul} % Highlights for text

% Standardize command font styles and environments
\newcommand{\doccmd}[1]{\texttt{\textbackslash#1}}% command name -- adds backslash automatically
\newcommand{\docopt}[1]{\ensuremath{\langle}\textrm{\textit{#1}}\ensuremath{\rangle}}% optional command argument
\newcommand{\docarg}[1]{\textrm{\textit{#1}}}% (required) command argument
\newcommand{\docenv}[1]{\textsf{#1}}% environment name
\newcommand{\docpkg}[1]{\texttt{#1}}% package name
\newcommand{\doccls}[1]{\texttt{#1}}% document class name
\newcommand{\docclsopt}[1]{\texttt{#1}}% document class option name
\newenvironment{docspec}{\begin{quote}\noindent}{\end{quote}}% command specification environment

\include{mathcommands.extratex}

\begin{document}

\maketitle% this prints the handout title, author, and date

\begin{abstract}
\noindent
We introduce the concept of extremal graph theory, starting with Turan's theorem. Then we introduce Szemerédi's regularity lemma as a tool for extremal graph theory.
\end{abstract}

\section{Extremal graphs}

We start with the central definition of extremal graph theory, and then we explain what it actually means through some exercises.

\begin{definition}
  Given any graph $H$, we say that a graph $G$ is \emph{$H$-free} if it has no subgraph isomorphic to $H$. We say that it is \emph{maximal $H$-free} if adding any edge to it would create a subgraph isomorphic to $H$, and we say that is is \emph{maximum $H$-free} (or \emph{extremal} among $H$-free graphs) if additionally no other $H$-free graph has more edges than $G$.

  For each integer $n$, we define the \emph{extremal function for $H$}, denoted $\ex(n; H)$, to be the number of edges of a maximum $H$-free graph on $n$ vertices.
\end{definition}

\begin{xca}
  As a warm-up exercise, if $H$ is the path on three vertices, what is $\ex(n;H)$? What are the extremal graphs for this problem?

  Letting $H_k$ be a star graph with $k$ leaves,\sidenote[][]{That is, a tree with one root with $k$ children, and no other vertices or edges.} what is $\ex(n;H_k)$? Which are the extremal graphs here?
\end{xca}

Having done this warmup, we can move on to the original question that motivated the start of extremal graph theory: How many edges can a graph have if it does not contain any triangles? This requirement clearly imposes \emph{some} bound on the number of edges -- a complete graph certainly contains a triangle -- but what is the bound?

\begin{xca}
  Letting $H = K_3$, the triangle graph, what is $\ex(n; K_3)$? What do the extremal graphs look like?\sidenote[][]{Side exercise: Can you find a graph that is maximal triangle-free but not extremal?}
\end{xca}

\begin{xca}
  Can you generalize what you just did to finding $\ex(n; K_k)$ for $k > 3$?
\end{xca}

%\bibliography{references}
%\bibliographystyle{plainnat}

\end{document}
